%!TEX program = xelatex
%!TEX spellcheck = en_GB
\documentclass[final]{article}
\input{../../.library/preamble.tex}
\input{../../.library/style.tex}
\addbibresource{../../.library/bibliography.bib}

\begin{document}
\section{Analysis}
As part of the code improvement process the profiling programs gprof and MCprof were used to investigate the importance of the different functions in the application.
Of course, because of the simplicity of the matrix multiplication application, the results of the profiling efforts do not offer anything more than trivial results.
This simple analysis therefore serves more as a step up to the more complicated analysis of the mean shift tracking application.
\Cref{app:profiling} contains the call graph generated by MCprof profiling for a matrix size of 16.
As can be expected for such a small matrix, the functions for measuring execution time take a significant part of the total execution time, whereas the multiplication itself makes up the rest.

The makefile of the matrix multiplication was also edited to allow for runtime profiling with gprof.
Gprof would only output profiling information once the matrix size was large enough (once there was actually something to measure) so the only relevant function in the gprof result was the now significantly more important matrix multiplication function.

To summarize: all optimisations should be focused on improving matrix multiplication performance, which happens to be the only function the application has.
\end{document}