%!TEX program = xelatex
%!TEX spellcheck = en_GB
\documentclass[final]{article}
\input{../../.library/preamble.tex}
\input{../../.library/style.tex}
\addbibresource{../../.library/bibliography.bib}

\begin{document}
\section{Analysis}
In order to determine what parts of the code were most important to improve, the basic application was profiled using MCProf and gprof. The results of the gprof profiling can be found in \cref{tab:gprof}.

\begin{table}[H]
\centering
\begin{tabular}{llllll}
name & \% time& self seconds & calls &  self ms/call & total ms/call \\
MeanShift::CalWeight & 34.90   &   0.15  &    157&     0.96 &    0.96  \\
MeanShift::track & 27.92    &   0.12   &    32 &    3.75  &  13.41 \\
MeanShift::pdf\_representation & 27.92     &  0.12    &  158  &   0.76   &  1.01  \\
MeanShift::Epanechnikov\_kernel & 9.31      &   0.04     & 158   &  0.25    & 0.25  \\
\end{tabular}
\caption{Gprof output for the basic application filtered on functions that are reported to last more than 0.00 seconds}
\label{tab:gprof}
\end{table}

Unsurprisingly, all the functions that take a significant amount of times are the functions needed for the tracking algorithm. The function that has the most significant impact on performance is the CalWeight function, but both pdf\_representation and the calculations in track also take a large amount of time. Therefore, those three functions should be the focus when improving the application.

\begin{figure}[H]
\includegraphics[width=\textwidth]{resources/callgraphAll.pdf}
\caption{Callgraph produced by MCProf}
\label{fig:callgraph}
\end{figure}

A callgraph produced by MCProf can be found in \cref{fig:callgraph}. Unfortunately, because MCProf only works on Intel processors it could not be run on the Beagleboard and was instead run on the server used for compilation. The difference in processing power between the server and the Beagleboard explains the difference between the gprof and MCProf profiling. The explanation for CalWeight's dominance in the callgraph can be found in \cref{tab:memprofile}. The CalWeight function has significantly more memory accesses than the other functions. Whereas the server is faster than the Beagleboard with computations, the communications overhead handicaps the obtainable speed-up in the CalWeight function. A possible decrease in the amount of necessary memory accesses in the CalWeight function will most likely cause a significant performance improvement.

\begin{table}[H]
\centering
\begin{tabular}{llll}
Name & Total & Reads & Writes\\
MeanShift::CalWeight   &  551678840  &   436089346   &  115589494\\
MeanShift::pdf\_representation    &  40448790  &    26733846   &   13714944\\
main   &    9293040   &    5895758    &  3397282\\
MeanShift::Epanechnikov\_kernel    &   3552009       & 252870  &     3299139\\
MeanShift::track   &    3429327   &    3407927      &   21400\\

\end{tabular}
\caption{Flat memory profile produced by MCProf}
\label{tab:memprofile}
\end{table}


\end{document}