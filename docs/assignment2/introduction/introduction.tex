%!TEX program = xelatex
%!TEX spellcheck = en_GB
\documentclass[final]{article}
\input{../../.library/preamble.tex}
\input{../../.library/style.tex}
\addbibresource{../../.library/bibliography.bib}
\begin{document}
\section{Introduction}
Interest in heterogeneous computing as a means of increasing computational performance, without relying on trivial improvements like increased clock speeds and more cores, has increased over the last few years.
Offloading processing to specialised hardware for specific tasks can help with attaining significant speed-ups.
For the course IN4342 -- Embedded Systems Laboratory, a matrix multiplication application and a mean shift tracking application are provided with a BeagleBoard to run them on.
These applications can be improved and significantly sped up by making optimal use of both the ARM and DSP on the BeagleBoard.
This short report discusses the results of the speed-up efforts for the matrix multiplication application, the smallest of the two programs.
First, the results of the profiling efforts will be discussed.
Then, multiple areas of possible improvement will be identified in the code and implemented.
Finally, the benchmarking results of the different implementations will be compared for multiple matrix sizes.
\end{document}